\documentclass[12pt,letterpaper]{article}
\usepackage[utf8]{inputenc}
\usepackage{amsmath}
\usepackage{amsfonts}
\usepackage{amssymb}
\usepackage{cancel}
\usepackage[hmargin=2cm,vmargin=2cm]{geometry}
\usepackage{graphicx}
\usepackage{setspace}

\newcommand{\proofend}{\mbox{ }\hfill$\Box$\\}
\newcommand{\ppf}[2]{\frac{\partial #1}{\partial #2}}
\newcommand{\ddf}[2]{\frac{\mathrm{d} #1}{\mathrm{d} #2}}
\newcommand{\ee}[1]{\cdot10^{#1}}

\doublespacing

\begin{document}

To use short electron pulses that emanate from a source, the pulse has to be delivered to a target at a different location. At the source, the pulse has an energy spread. The dispersion that arises from this initial energy spread limits the resolution at the target. The basic idea of an electron dispersion compensator is to alter the electron paths to compensate for the energy differences present in the electron pulse.
\end{document}
