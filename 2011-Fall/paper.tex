\documentclass[12pt,letterpaper]{article}
\usepackage[utf8]{inputenc}
\usepackage{amsmath}
\usepackage{amsfonts}
\usepackage{amssymb}
\usepackage{cancel}
\usepackage[hmargin=2cm,vmargin=2cm]{geometry}
\usepackage{graphicx}
\usepackage{setspace}

\newcommand{\proofend}{\mbox{ }\hfill$\Box$\\}
\newcommand{\pdf}[2]{\frac{\partial #1}{\partial #2}}
\newcommand{\ddf}[2]{\frac{\mathrm{d} #1}{\mathrm{d} #2}}
\newcommand{\ee}[1]{\cdot10^{#1}}

\doublespacing

\begin{document}
\section{Introduction}
\subsection{Idea History}

To use short electron pulses that emanate from a source, the pulse has to be delivered to a target at a different location. At the source, the pulse has an energy spread. The dispersion that arises from this initial energy spread limits the resolution at the target. The basic idea of an electron dispersion compensator is to alter the electron paths to compensate for the energy differences present in the electron pulse.

Like the angled gratings in an optical dispersion compensator, the electron dispersion compensator uses a pair of magnetic fields to spatially separate the particles according to their energy. A delay can be applied to electrons spatially increasing the time of flight for higher energy particles, decreasing it for lower energy particles. We do this by sending the pulse through a pair of crossed magnetic and electric fields. The electric field sets up a linear potential. Higher energy electrons pass through a higher potential and lower energy particles pass through a lower potential. With the correct delay a 
, borrowed from the optical dispersion compensator,
\end{document}

